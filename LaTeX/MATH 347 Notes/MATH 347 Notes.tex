\documentclass[12pt, letterpaper]{article}

\usepackage[utf8]{inputenc}
\usepackage[framemethod=TikZ]{mdframed}
\usepackage[hidelinks]{hyperref}
\usepackage{mathtools, amssymb, amsmath, cleveref, fancyhdr, geometry, tcolorbox, graphicx, float, subfigure, arydshln, url, setspace, framed, pifont, physics, ntheorem, utopia, dsfont}
%%% for coding %%%
\usepackage{listings}
\usepackage[ruled, vlined, linesnumbered]{algorithm2e}

\geometry{letterpaper, left=2cm, right=2cm, bottom=2cm, top=2cm}

\pagestyle{fancy}
\fancyhead{}
\fancyhead[L]{\leftmark}
\fancyhead[R]{\rightmark}
\fancyfoot{}
\fancyfoot[C]{\thepage}
%\renewcommand{\headrulewidth}{0pt}
\renewcommand{\footrulewidth}{0pt}

\hypersetup{
	colorlinks = true,
	bookmarks = true,
	bookmarksnumbered = true,
	pdfborder = 001,
	linkcolor = blue
}

\definecolor{grey}{rgb}{0.49,0.38,0.29}
\definecolor{mygreen}{rgb}{0,0.6,0}


%%% for coding %%%
\lstset{basicstyle = \ttfamily\small,commentstyle = \color{mygreen}\textit, deletekeywords = {...}, escapeinside = {\%*}{*)}, frame = single, framesep = 0.5em, keywordstyle = \bfseries\color{blue}, morekeywords = {*}, emph = {self}, emphstyle=\bfseries\color{red}, numbers = left, numbersep = 1.5em, numberstyle = \ttfamily\small\color{grey},  rulecolor = \color{black}, showstringspaces = false, stringstyle = \ttfamily\color{purple}, tabsize = 4, columns = flexible}


\newcounter{index}[subsection]
\setcounter{index}{0}
\newenvironment*{df}[1]{\par\noindent\textbf{Definition \thesubsection.\stepcounter{index}\theindex\ (#1).}}{\par}

\newenvironment*{eg}{\begin{framed}\par\noindent\textbf{Example \thesubsection.\stepcounter{index}\theindex}}{\par\end{framed}}

\newenvironment*{thm}[1]{\begin{tcolorbox}\par\noindent\textbf{Theorem \thesubsection.\stepcounter{index}\theindex\ #1} \par}{\par\end{tcolorbox}}

\newenvironment*{cor}[1]{\par\noindent\textbf{Corollary \thesection.\stepcounter{index}\theindex\ #1:}}{\par}
\newenvironment*{lem}[1]{\par\noindent\textbf{Lemma \thesection.\stepcounter{index}\theindex\ #1:}}{\par}
\newenvironment*{ax}[1]{\par\noindent\textbf{Axiom \thesection.\stepcounter{index}\theindex\ #1:}}{\par}
\newenvironment*{prop}[1]{\par\noindent\textbf{Proposition \thesection.\stepcounter{index}\theindex\ #1:}}{\par}
\newenvironment*{conj}[1]{\par\noindent\textbf{Conjecture \thesection.\stepcounter{index}\theindex\ #1:}}{\par}
\newenvironment*{nota}{\par\noindent\textbf{Notation \thesection.\stepcounter{index}\theindex.}}{\par}

\newcounter{nprf}[subsection]
\setcounter{nprf}{0}
\newenvironment*{prf}{\par\indent\textbf{\textit{Proof \stepcounter{nprf}\thenprf.}}}{\hfill$\blacksquare$\par}
\newenvironment*{dis}{\par\indent\textbf{\textit{Disproof \stepcounter{nprf}\thenprf.}}}{\hfill$\blacksquare$\par}
\newenvironment*{sol}{\par\indent\textbf{\textit{Solution \stepcounter{nprf}\thenprf.}}\par}{\hfill{$\square$}\par}

\newtheorem{hint}{Hint}[section]
\newtheorem{rmk}{Remark}[section]
\newtheorem{ext}{Extension}[section]

\linespread{1.25}

\newcommand\perm[2][n]{_{#1}P_{#2}}
\newcommand\com[2][n]{_{#1}C_{#2}}

\def\Z{\mathbb{Z}}
\def\R{\mathbb{R}}
\def\C{\mathbb{C}}
\def\Q{\mathbb{Q}}
\def\N{\mathbb{N}}
\def\d{\mathrm{d}}
\def\1{\mathds{1}}
\def\epsilon{\varepsilon}
\def\emptyset{\varnothing}
\def\phi{\varphi}
\def\dsst{\displaystyle}
\def\st{\ s.t.\ }
\def\bar{\overline}
\def\E{\vb{E}}
\def\B{\vb{B}}
\def\L{\vb{L}}
\def\I{\vb{I}}
\def\Var{\vb{Var}}
\def\Cov{\vb{Cov}}
\def\MSE{\vb{MSE}}
\def\P{\vb{P}}
\def\M{\vb{M}}
\def\iid{i.i.d.}
\def\argmax{\arg\max}
\def\argmin{\arg\min}
\def\l{\ell}
\def\hat{\widehat}
\def\independ{\perp\!\!\!\perp}
\def\depend{\leftrightsquigarrow}

\title{Emory University\\\textbf{MATH 347 Non Linear Optimization}\\ Learning Notes}
\author{Jiuru Lyu}
\date{\today}

\begin{document}
\maketitle

\tableofcontents
\newpage

\section{Math Preliminaries}
\subsection{Introduction to Optimization}
\begin{df}{Optimization Problem}
	The main optimization problem can be stated as follows \begin{equation}\label{eq1}\min_{x\in S}f(x),\end{equation} where 
	\begin{itemize}
		\item $x$ is the \textit{optimization variable},
		\item $S$ is the \textit{feasible set}, and 
		\item $f$ is the \textit{objective function}.
	\end{itemize}
\end{df}
\begin{rmk}
	$\dsst\max_{x\in S}f(x)=-\min_{x\in S}-f(x)$. Hence, we will only study minimization problems.
\end{rmk}
\begin{thm}{Solving an Optimization Problem}
	\begin{itemize}
		\item Theoretical Analysis: analytic solution
		\item Numerical solution/optimization
	\end{itemize}	
\end{thm}
\begin{df}{Solution Methods depend on the type of $x$, $S$, and $f$}
	\begin{itemize}
		\item When $x$ is continuous (e.g., $\R$, $\R^n$, $\R^{m\times n}$, $\dots$), then the optimization problem stated in Eq. (\ref{eq1}) is a \textit{continuous optimization problem}. $_\textit{It will also be the focus of this class.}$ \par Opposite to continuous optimization problems, we have \textit{discrete optimization problem} if $x$ is discrete. \par If $x$ has both types of components, then we call the problem \textit{mixed}.
		\item Depending on $S$, we can have 
		\begin{itemize}
			\item \textit{Unconstrained problems}: where $S=\R^n$, $S=\R^{m\times n}$, $\dots$ ($m,n$ are fixed).
			\item \textit{Constrained problems}: where $S\subsetneq\R^n$, $S\subsetneq\R^{m\times n}$, $\dots$. \par $_\textit{Both types of problems will be studied.}$
		\end{itemize}
		\item Depending on $f$, we have 
		\begin{itemize}
			\item \textit{Smooth optimization problems}: $f$ has first and/or second order derivatives. \par $_\textit{Only smooth optimization problems will be studied.}$
			\item \textit{Non-smooth optimization problems}: $f$ is not differentiable. 
		\end{itemize}
	\end{itemize}	
\end{df}

\begin{df}{Linear Optimization/Program}
	If $f$ is linear and $S$ consists of linear constrains, then the optimization problem is called a \textit{linear problem/program}.	
\end{df}
\begin{eg}{Classification of Optimization Problems}
	\begin{enumerate}
		\item Consider the following problem \[\min_{x_1,x_2,x_3}x_1^2-4x_1x_2+3x_2x_3+\sin{x_3}\]
		\begin{sol}
			\begin{itemize}
				\item Optimization variable: $x=(x_1,x_2,x_3)\in\R^3$. $\longrightarrow$ continuous.
				\item Feasible set: $S=\R^3$. $\longrightarrow$ unconstrained.
				\item Objective function: $f(x_1,x_2,x_3)=x_1^2-4x_1x_2+3x_2x_3+\sin{x_3}$. $\longrightarrow$ smooth but non-linear. 
			\end{itemize}	
		\end{sol}
		\item Consider the following problem \[\max_{\substack{4x_1+7x_2+3x_3\leq1 \\ x_1,x_2,x_3\geq0}}x_1+2x_2+3x_3\]
		\begin{sol}
			\begin{itemize}
				\item Optimization variable: $x=(x_1,x_2,x_3)\in\R^3$. $\longrightarrow$ continuous.
				\item Feasible set: $S=\qty{(x_1,x_2,x_3):x_1,x_2,x_3\geq0, 4x_1+7x_2+3x_3\leq 1}\subsetneq\R^3$. $\qquad\longrightarrow$ constrained.
				\item Objective function: $f(x_1,x_2,x_3)=x_1+2x_2+3x_3$. $\longrightarrow$ smooth and linear. 
			\end{itemize}	
		\end{sol}
		\begin{rmk}
			This problem can be considered as the budget constrained optimization problem in Economics. 	
		\end{rmk}

		\item Consider the following problem \[\min_{x_1,x_2\geq0}4x_1-3\qty|x_2|+\sin(x_1^2-2x_2)\]
		\begin{sol}
			\begin{itemize}
				\item Optimization variable: $x=(x_1,x_2)\in\R^2$. $\longrightarrow$ continuous. 
				\item Feasible set: $S=\qty{(x_1,x_2):x_1,x_2\geq0}\subsetneq\R62$. $\longrightarrow$ constrained.
				\item Objective function: $f(x_1,x_2)=4x_1-3\qty|x_2|+\sin(x_1^2-2x_2)$. $\longrightarrow$ non-smooth and non-linear.
			\end{itemize}	
		\end{sol}
		\begin{rmk}
			In this particular problem, $x_2\geq0$, and so $f(x_1,x_2)=4x_1-3x_2+\sin\qty(x_1^2-2x_2)$ on the feasible set. Hence, this problem can be equivalently written as \[\min_{x_1,x_2\geq0}4x_1-3x_2+\sin\qty(x_1^2-2x_2),\] which is a smooth optimization problem. 
		\end{rmk}
	\end{enumerate}
\end{eg}


\newpage
\section{Unconstrained Optimization}

\newpage
\section{Least Square}

\newpage
\section{Constrained Optimization}

\end{document}