\documentclass[12pt, addpoints]{exam}
\usepackage[utf8]{inputenc}
\usepackage[framemethod=TikZ]{mdframed}
\usepackage[hidelinks]{hyperref}
\usepackage{mathtools, amssymb, amsmath, cleveref, geometry, tcolorbox, graphicx, float, subfigure, arydshln, url, setspace, framed, pifont, physics, ntheorem, color, utopia}
%%% for coding %%%
\usepackage{listings}
\usepackage[ruled, vlined, linesnumbered]{algorithm2e}

\newcommand{\tf}[1][{}]{%
\fillin[#1][0.25in]%
}
\setlength\dottedlinefillheight{.25in}
\colorfillwithdottedlines
\colorfillwithlines
\definecolor{FillWithLinesColor}{gray}{0.5}
\definecolor{FillWithDottedLinesColor}{gray}{0.5}

\pagestyle{headandfoot}
\firstpageheader{}
{Wellesley College\\
Final Examination, Spring 1993\\
Mathematics 115\\
}
{\large\bfseries Name:\enspace\makebox[1.5in]{\hrulefill}}
\runningheader{Mathematics 115 (Continued)}{}{Spring, 1993}
\footer{}
{Page \thepage\ of \numpages}
{\iflastpage{End of exam.}{Please go on to the next page\ldots}}

\printanswers
\begin{document}
\pagestyle{headandfoot}
\begin{coverpages}
\title{This is the title}
\maketitle
\vspace{3in}
\begin{center}
\fbox{\fbox{\parbox{5.5in}{\centering
Answer the questions in the spaces provided on the
question sheets. If you run out of room for an answer,
continue on the back of the page.}}}	
\end{center}
\vspace{0.5in}\hfill\large\bfseries Signature:\enspace\makebox[1.5in]{\hrulefill}\newline
\vspace{1in}
\hfill\large\bfseries Date:\enspace\makebox[1.5in]{\hrulefill}
\end{coverpages}

\newpage 
\begin{center}
\fbox{\fbox{\parbox{5.5in}{\centering
Answer the questions in the spaces provided on the
question sheets. If you run out of room for an answer,
continue on the back of the page.}}}	
\end{center}
\vspace{0.1in}
\makebox[\textwidth]{Name and section:\enspace\hrulefill}
\vspace{0.2in}
\makebox[\textwidth]{Instructor’s name:\enspace\hrulefill}
\vspace{1.5in}
\begin{center}
	\gradetable[v][questions]
\end{center}
\begin{center}
	\gradetable[h][questions]
\end{center}

\newpage
\uplevel{Questions \ref{exact-start} through~\ref{exact-end} should
be evaluated completely, not just approximated.}
\begin{questions}
	\question[10]\label{exact-start} One of these things is not like the others; one of these things is not the same. Which one is different?
	\begin{choices}
		\choice John
		\choice Paul
		\choice George
		\CorrectChoice Ringo
	\end{choices}	
	\question[15] How much wood would a woodchuck chuck if a woodchuck could chuck wood?
	\begin{oneparchoices}
		\choice John
		\choice Paul
		\choice George
		\CorrectChoice Ringo
	\end{oneparchoices}	
	\question[10] Compute $\displaystyle\int_0^1 x^2 \, dx$.
	\begin{checkboxes}
		\choice John
		\choice Paul
		\choice George
		\CorrectChoice Ringo
	\end{checkboxes}
	\question[10] \fillin[black] is the color of my true love’s hair
	\question[10]\label{exact-end} True/False
	\begin{parts}
		\part \tf[T] The world is all that is the case.
		\part \tf[F] My favorite color is blue.	
	\end{parts}
	\question[10] Why is there air?
	\vspace{2in}
	\answerline[Tom]
	\question What if there were no air?
	\begin{parts}
		\part[5] Describe the effect on the balloon industry. 
		\makeemptybox{1in}
		\part[5] Describe the effect on the aircraft industry.
		\fillwithlines{1in}
		\fillwithdottedlines{1in}
		\smallskip\fillwithgrid{1in}
	\end{parts}
	\question
	\begin{parts}
		\part [10] Define the universe. Give three examples.
		\begin{solutionbox}{2in}
		Once upon a midnight dreary, while I pondered, weak and weary, Over many a quaint and curious volume of forgotten lore--- While I nodded, nearly napping, suddenly there came a tapping, As of some one gently rapping, rapping at my chamber door. ‘‘\,’Tis some visitor,’’ I muttered, ‘‘tapping at my chamber door--- Only this and nothing more.’’
		\end{solutionbox}
		\part If the universe were to end, how would you know?
		\begin{subparts}
			\subpart[5] Justify your answers with real world examples.
			\subpart Prove your answers.
			\begin{subsubparts}
				\subsubpart[2\half] Think of the greek letters.
				\subsubpart[2\half] This is fun.
			\end{subsubparts}
		\end{subparts}
	\end{parts}
	\vspace{3in}
	\bonusquestion[5] This is a bonus for you.
\end{questions}
\end{document}