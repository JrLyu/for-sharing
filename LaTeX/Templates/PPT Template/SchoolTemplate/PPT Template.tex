\documentclass[aspectratio=43, 9pt, utf8, mathserif]{beamer}

\usepackage{amssymb, amsmath, amsthm, amsfonts, epsfig, mathtools, times, verbatim, anyfontsize, subfigure, graphicx, tabularx, tikz, mathptmx, physics, utopia, pgf, beamerfoils}
\usepackage[export]{adjustbox}
\setbeamertemplate{caption}[numbered]
\newcounter{saveenumi}
\resetcounteronoverlays{saveenumi}

%% bibtex Settings
\usepackage[backend=bibtex,style=numeric-comp,sorting=none]{biblatex} %不列出所有作者
%\usepackage[backend=bibtex,sorting=none,maxnames=9,minnames=3]{biblatex} %列出所有作者
% \addbibresource{ref.bib} %BibTeX数据文件及位置
\setbeamerfont{footnote}{size=\tiny} %设置脚注引用文献的字体大小
\setbeamertemplate{bibliography item}[text] %设置参考文献图标样式数字标号

%% For coding
\usepackage{listings} 
\lstset{
	frame=shadowbox, rulesepcolor=\color{red!20!green!20!blue!20}, keywordstyle=\color{blue}\bfseries, commentstyle=\color{orange}\ttfamily, 	backgroundcolor=\color{darkgray!6}, showstringspaces=false, numbers=left, numberstyle=\tiny, basicstyle=\ttfamily, stringstyle=\ttfamily, breaklines=true, extendedchars=false,  texcl=true, morekeywords={classdef,function,global,parfor,persistent,spmd,plot}} 
	
% Theme selection
\usetheme{Boadilla}

% Colors and Frames
\useinnertheme{circles}
\useoutertheme{miniframes}
\usecolortheme{beaver}
\usefonttheme{serif}

\definecolor{emoryblue}{RGB}{1, 33, 105} 
\definecolor{lightblue}{RGB}{0, 125, 186}
\definecolor{mediumblue}{RGB}{ 0, 51, 160}
\definecolor{darkblue}{RGB}{12, 35, 64}
\definecolor{red}{RGB}{185, 58, 38}
\definecolor{green}{RGB}{72, 127, 132}
\definecolor{gray1}{RGB}{217, 217, 214}
\definecolor{gray5}{RGB}{177, 179, 179}
\definecolor{gray3}{RGB}{208, 208, 206}

% miniframes mode
\setbeamertemplate{itemize items}{\color{darkblue}$\bullet$} %bulletpoints
\setbeamercolor{section number projected}{bg = mediumblue!80, fg=gray1}% toc
\setbeamercolor{title}{bg=emoryblue!80!darkblue!80,fg=gray1} % titlepage
\setbeamercolor{palette secondary}{bg=darkblue!90, fg=gray1} % left
\setbeamercolor{palette tertiary}{bg=emoryblue!30,fg=darkblue} % middle
\setbeamercolor{palette primary}{bg=gray3,fg=darkblue} % right
% Title page
\setbeamercolor*{author}{fg = darkblue}
\setbeamercolor*{institute}{fg = darkblue}
\setbeamercolor*{date}{fg = darkblue}
% subtitles (headings)
\setbeamercolor*{titlelike}{fg=darkblue}
% blocks
\setbeamercolor*{structure}{bg=darkblue!20, fg=darkblue}
\setbeamercolor*{example text}{use=structure, fg = green}
\setbeamercolor*{alerted text}{use=structure, fg = red}
% Logo
\MyLogo{\pgfputat{\pgfxy(-0.5,0)}{\pgfbox[right,base]{\includegraphics[height=2cm]{fig/logo}}}}

%% ------- Information on the Presentation ------
\author[My name]{Jiuru Lyu}
\title[Name to be Include]{Title of the Presentation}
\institute[My institution]{Full name of my institution}
\date{\today}

%% ------ Start ------
\begin{document}
\everymath{\displaystyle}

% Title page
\LogoOn
\begin{frame}
    \titlepage 
\end{frame}
\LogoOff

% Toc Page
\begin{frame}
    \frametitle{\textbf{Table of Contents}}
    \tableofcontents % 生成目录(如果为空,请编译两次)
\end{frame}

\section{Introduction}\label{sec:introduction}
\begin{frame}
\frametitle{\textbf{Introduction}}
\begin{block}{Block Title 1}
\begin{itemize}
    \item Lorem ipsum dolor sit amet, consectetur adipiscing elit, sed do eiusmod tempor incididunt ut labore . 
    \item Ut enim ad minim veniam, quis nostrud exercitation ullamco laboris nisi ut aliquip ex ea commodo consequat.
    \item Duis aute irure dolor in reprehenderit in voluptate velit esse cillum dolore eu fugiat nulla pariatur.  \end{itemize}
\end{block}
    
\begin{exampleblock}{Block Title 2}
\begin{enumerate}
    \item Lorem ipsum dolor sit amet, consectetur adipiscing elit, sed do eiusmod tempor incididunt ut labore . 
    \item Ut enim ad minim veniam, quis nostrud exercitation ullamco laboris nisi ut aliquip ex ea commodo consequat.
    \item Duis aute irure dolor in reprehenderit in voluptate velit esse cillum dolore eu fugiat nulla pariatur.  \end{enumerate}
\end{exampleblock}
\end{frame}

\begin{frame}
\frametitle{\textbf{Example of subfigure}}
\centering
Idea A $\Longleftrightarrow$ Idea B
\vskip 2em
\begin{figure}
    \subfigure[Image Caption]{\includegraphics[width=0.4\linewidth]{example-image}}
    \subfigure[Image Caption]{\includegraphics[width=0.4\linewidth]{example-image}}
    \caption{This is a caption}
\end{figure}
\end{frame}

\begin{frame}
\frametitle{\textbf{Black hole}}
\begin{exampleblock}{The metric and the electromagnetic field of the spherically symmetric solution}
\begin{align}
    ds^2&=-fdt^2+\frac{dr^2}{f}+r^2 d\Omega_2^2\,,\label{BImetric}\\
    F&=Edt\wedge dr\label{BIE}\,, \quad E=\frac{Q}{\sqrt{r^4+Q^2/b^2}}\,.
\end{align}
\end{exampleblock}
where
\begin{equation}
\begin{split}
    f=&1-\frac{2M}{r}+\frac{r^2}{l^2}+\frac{2b^2}{r}\int_r^\infty \Bigl(\sqrt{r^4+\frac{Q^2}{b^2}}-r^2\Bigr)dr\nonumber\\
    =&1-\frac{2M}{r}+\frac{r^2}{l^2}+\frac{2b^2 r^2}{3}\Bigl(1-\sqrt{1+\frac{Q^2}{b^2 r^4}}\Bigr)\nonumber\\
    &+\frac{4Q^2}{3r^2}\,{}_2 F_1\left(\frac{1}{4},\frac{1}{2}; \frac{5}{4};-\frac{Q^2}{b^2 r^4}\right)\,,
\end{split}
\end{equation}
and $_2 F_1$ is the hypergeometry function, $M$ and $Q$ stand for black hole mass and charge. $d\Omega$ is the unit sphere on $S^2$.
\end{frame}

\section{Content}\label{sec:content}
\begin{frame}
\frametitle{\textbf{Content}}
\begin{block}{Mass $M$}
\begin{equation}
    f(r_h) = 0 \Longrightarrow M = \frac{T}{v}-\frac{1-\sqrt{\frac{16}{v^4}+1}}{4 \pi }-\frac{1}{2 \pi  v^2}
\end{equation}
\end{block}
    
\begin{exampleblock}{Hawking temperature $T$}
\begin{equation}\label{eq:T}
    T = f^{\prime}(r_+) / 4 \pi = \frac{1}{4\pi r_+}\!\left[1\!+\!\frac{3r_+^2}{l^2}\!+\!2b^2 r_+^2\Bigl(1\!-\!\sqrt{1\!+\!\frac{Q^2}{b^2 r_+^4}}\Bigr)\right]
\end{equation}
\end{exampleblock}
    
\begin{alertblock}{Electric potential $\Phi$}
\begin{equation}
    \Phi=\int_{r_+}^\infty E dr
    =\frac{Q}{r_+}\,{}_2 F_1\!\left(\frac{1}{4},\frac{1}{2};\frac{5}{4};-\frac{Q^2}{b^2 r_+^4}\right)\,.
\end{equation}
\end{alertblock}
    
The corresponding entropy is $S = \pi r_+^2$, The specific volume $v = 2 r_+ l_P^2$ and corresponding pressure $P = - \frac{\Lambda}{8 \pi} = \frac{3}{8 \pi l^2}$
\end{frame}

\section{Conclusion}\label{sec:conclusion}
\begin{frame}
\frametitle{Conclusion}
\begin{block}{Conclusion 1}
    Lorem ipsum dolor sit amet, consectetur adipiscing elit, sed do eiusmod tempor incididunt ut labore et dolore magna aliqua.
    Ut enim ad minim veniam, quis nostrud exercitation ullamco laboris nisi ut aliquip ex ea commodo consequat.
\end{block}

\begin{exampleblock}{Conclusion 2}
    Lorem ipsum dolor sit amet, consectetur adipiscing elit, sed do eiusmod tempor incididunt ut labore et dolore magna aliqua.
    Ut enim ad minim veniam, quis nostrud exercitation ullamco laboris nisi ut aliquip ex ea commodo consequat. 
\end{exampleblock}
\end{frame}

\LogoOn
\begin{frame}[noframenumbering]
\centering
\fontsize{40}{50}\selectfont Q \& A\\Thank You!
\end{frame}
\LogoOff

\end{document}