\documentclass[12pt, letterpaper]{article}

\usepackage[utf8]{inputenc}
\usepackage[framemethod=TikZ]{mdframed}
\usepackage[hidelinks]{hyperref}
\usepackage{mathtools, amssymb, amsmath, cleveref, fancyhdr, geometry, tcolorbox, graphicx, float, subfigure, arydshln, url, setspace, framed, pifont, physics, ntheorem, utopia, dsfont}
%%% for coding %%%
\usepackage{listings}
\usepackage[ruled, vlined, linesnumbered]{algorithm2e}

\geometry{letterpaper, left=2cm, right=2cm, bottom=2cm, top=2cm}

\pagestyle{fancy}
\fancyhead{}
\fancyhead[L]{\leftmark}
\fancyhead[R]{\rightmark}
\fancyfoot{}
\fancyfoot[C]{\thepage}
%\renewcommand{\headrulewidth}{0pt}
\renewcommand{\footrulewidth}{0pt}

\hypersetup{
	colorlinks = true,
	bookmarks = true,
	bookmarksnumbered = true,
	pdfborder = 001,
	linkcolor = blue
}

\definecolor{grey}{rgb}{0.49,0.38,0.29}
\definecolor{mygreen}{rgb}{0,0.6,0}


%%% for coding %%%
\lstset{basicstyle = \ttfamily\small,commentstyle = \color{mygreen}\textit, deletekeywords = {...}, escapeinside = {\%*}{*)}, frame = single, framesep = 0.5em, keywordstyle = \bfseries\color{blue}, morekeywords = {*}, emph = {self}, emphstyle=\bfseries\color{red}, numbers = left, numbersep = 1.5em, numberstyle = \ttfamily\small\color{grey},  rulecolor = \color{black}, showstringspaces = false, stringstyle = \ttfamily\color{purple}, tabsize = 4, columns = flexible}


\newcounter{index}[subsection]
\setcounter{index}{0}
\newenvironment*{df}[1]{\par\noindent\textbf{Definition \thesubsection.\stepcounter{index}\theindex\ (#1).}}{\par}

\newenvironment*{eg}{\begin{framed}\par\noindent\textbf{Example \thesubsection.\stepcounter{index}\theindex}}{\par\end{framed}}

\newenvironment*{thm}[1]{\begin{tcolorbox}\par\noindent\textbf{Theorem \thesubsection.\stepcounter{index}\theindex\ #1} \par}{\par\end{tcolorbox}}

\newenvironment*{cor}[1]{\par\noindent\textbf{Corollary \thesection.\stepcounter{index}\theindex\ #1:}}{\par}
\newenvironment*{lem}[1]{\par\noindent\textbf{Lemma \thesection.\stepcounter{index}\theindex\ #1:}}{\par}
\newenvironment*{ax}[1]{\par\noindent\textbf{Axiom \thesection.\stepcounter{index}\theindex\ #1:}}{\par}
\newenvironment*{prop}[1]{\par\noindent\textbf{Proposition \thesection.\stepcounter{index}\theindex\ #1:}}{\par}
\newenvironment*{conj}[1]{\par\noindent\textbf{Conjecture \thesection.\stepcounter{index}\theindex\ #1:}}{\par}
\newenvironment*{nota}{\par\noindent\textbf{Notation \thesection.\stepcounter{index}\theindex.}}{\par}

\newcounter{nprf}[subsection]
\setcounter{nprf}{0}
\newenvironment*{prf}{\par\indent\textbf{\textit{Proof \stepcounter{nprf}\thenprf.}}}{\hfill$\blacksquare$\par}
\newenvironment*{dis}{\par\indent\textbf{\textit{Disproof \stepcounter{nprf}\thenprf.}}}{\hfill$\blacksquare$\par}
\newenvironment*{sol}{\par\indent\textbf{\textit{Solution \stepcounter{nprf}\thenprf.}}\par}{\hfill{$\square$}\par}

\newtheorem{hint}{Hint}[section]
\newtheorem{rmk}{Remark}[section]
\newtheorem{ext}{Extension}[section]

\linespread{1.25}

\newcommand\perm[2][n]{_{#1}P_{#2}}
\newcommand\com[2][n]{_{#1}C_{#2}}

\def\Z{\mathbb{Z}}
\def\R{\mathbb{R}}
\def\C{\mathbb{C}}
\def\Q{\mathbb{Q}}
\def\N{\mathbb{N}}
\def\d{\mathrm{d}}
\def\1{\mathds{1}}
\def\epsilon{\varepsilon}
\def\emptyset{\varnothing}
\def\phi{\varphi}
\def\dsst{\displaystyle}
\def\st{\ s.t.\ }
\def\bar{\overline}
\def\E{\vb{E}}
\def\B{\vb{B}}
\def\L{\vb{L}}
\def\I{\vb{I}}
\def\Var{\vb{Var}}
\def\Cov{\vb{Cov}}
\def\MSE{\vb{MSE}}
\def\P{\vb{P}}
\def\M{\vb{M}}
\def\iid{i.i.d.}
\def\argmax{\arg\max}
\def\argmin{\arg\min}
\def\l{\ell}
\def\hat{\widehat}
\def\independ{\perp\!\!\!\perp}
\def\depend{\leftrightsquigarrow}

\title{Emory University\\\textbf{QTM 220 Regression Analysis}\\ Learning Notes}
\author{Jiuru Lyu}
\date{\today}

\begin{document}
\maketitle

\tableofcontents

\newpage
\section{Descriptive Statistics and Binary Covariates}
\begin{df}{Location}
	The \textit{location} of the data is where it is. It is about approximating the data by a constant. \[Y_i\approx\mu,\quad\text{for }i=1,\dots,n\]	
\end{df}
\begin{eg}
	Different ways to summarize location: mean, median
\end{eg}
\begin{df}{Spread}
	The \textit{spread} of the data is how far it tends to be from is location. 	
\end{df}
\begin{df}{Residuals}
	Spread summarizes the size of the \textit{residuals} left over after constant approximation. We use $\hat\epsilon$ to denote residuals. \[\residual_i\coloneqq Y_i-\hat\mu.\]
\end{df}
\begin{df}{Median Absolute Deviation and Standard Deviation}
	\begin{itemize}
		\item The \textit{median absolute deviation (MAD)} is the median size of residuals.
		\item The \textit{standard deviation (sd)} is the square root of the mean squared size of residuals. 
		\begin{rmk}The standard deviation is a sort of average in which big residuals count more than smaller ones. \end{rmk}
	\end{itemize}
\end{df}
\begin{df}{Distribution}
	We use \textit{histograms} to summarize the \textit{distribution} of the data. 
\end{df}
\begin{rmk}
	Distribution of the data tells us more information than location and spread, but less than dot plot. \emph{For example, in this context, dot plot also include the identities of the individuals in addition to the number of people having salary in the range. }
\end{rmk}
\begin{df}{Binary Data}
	\textit{Binary data} only have two options, and we usually denote those two options as $1$'s and $0$'s. 	
\end{df}
\begin{cor}{}
	Hence, when drawing a dot plot, everyone falls into either of the two lines representing $1$ and $0$.
\end{cor}
\begin{thm}{Location of Binary Data}
	The median is whichever outcome is the most common, and the mean is the proportion of $1$'s in the data. 
\end{thm}
\begin{rmk}
	Hence, a histogram tells us no more information than $\hat\mu$.	
\end{rmk}
\begin{thm}{Spread of Binary Data}
	\begin{itemize}
		\item Median absolute deviation will always be $0$ in a binary case.
		\item The standard deviation is the square root of the mean squared distance from the mean, and \[\text{sd}=\sqrt{\hat\mu\qty(1-\hat\mu)}.\]
	\end{itemize}
\end{thm}
\begin{prf}
	The claim concerning MAD is trivial. \textit{Hint: there's only two possible values in the data, so median and MAD should always be the same.}\par Now, let's consider the claim on standard deviation. \begin{align*}\sd^2&=\dfrac{1}{n}\sum_{i=1}^n\qty(Y_i-\hat\mu)^2\\&=\dfrac{1}{n}\sum_{y:\qty{0,1}}\sum_{i:Y_i=y}\qty(Y_i-\hat\mu)^2\\&=\dfrac{1}{n}\qty{N_1\qty(1-\hat\mu^2)+\qty(n-N_1)\qty(0-\hat\mu^2)}&[N_1=\text{number of }1\text{'s}]\\&=\dfrac{1}{n}\qty{N_1\qty(1-2\hat\mu+\hat\mu^2)+\qty(n-N_1)\hat\mu^2}\\&=\dfrac{1}{n}\qty{N_1-2N_1\hat\mu+n\hat\mu^2}\\&=\dfrac{1}{n}\qty{n\hat\mu-2n\hat\mu\cdot\hat\mu+n\hat\mu^2}&[N_1=n\hat\mu]\\&=\dfrac{1}{n}\qty{n\hat\mu-n\hat\mu^2}\\&=\hat\mu-\hat\mu^2=\hat\mu\qty(1-\hat\mu).\end{align*} Therefore, we know \[\sd=\sqrt{\hat\mu\qty(1-\hat\mu)}.\]
\end{prf}
\begin{rmk}
	In binary data, knowing the mean $\equiv$ knowing everything else. 
\end{rmk}



\end{document}