\documentclass[12pt,a4paper]{article}

\usepackage[utf8]{inputenc}
\usepackage[T1]{fontenc}
\usepackage{lmodern}
\usepackage{mathtools}
\usepackage{amssymb}
\usepackage{ntheorem}
\usepackage[framemethod=TikZ]{mdframed}
\usepackage{amsmath}
\usepackage[hidelinks]{hyperref}
\usepackage[most]{tcolorbox}
\usepackage{fancyhdr}
\usepackage{geometry}
\usepackage{graphicx}
\usepackage{float} 
\usepackage{subfigure} 
\usepackage{arydshln}
\usepackage{url}
\usepackage{setspace}
\usepackage{mathptmx}
\usepackage{framed}
\usepackage{xcolor}
\usepackage{tikz}

\geometry{a4paper, left=2cm, right=2cm, bottom=2cm, top=2cm}

\pagestyle{fancy}
\fancyhead{}
\fancyhead[L]{\leftmark}
\fancyhead[R]{\rightmark}
\fancyfoot{}
\fancyfoot[C]{\thepage}
%\renewcommand{\headrulewidth}{0pt}
\renewcommand{\footrulewidth}{0pt}

\definecolor{blue}{rgb}{0,0.45,1.14}
\definecolor{red}{rgb}{0.77,0.12,0.23}
\definecolor{grey}{rgb}{0.49,0.38,0.29}
\definecolor{green}{rgb}{0,0.42,0.24}
\definecolor{orange}{rgb}{2.07,0.69,0.32}

\theorembodyfont{\upshape}
\theoremseparator{.}
\newtheorem{thm}{Theorem}[subsection]
\newtheorem{df}{Definition}[subsection]
\newtheorem{eg}{Example}[subsection]
\newenvironment*{sol}{\par\indent\textbf{\textit{Solution. }}}{\hfill{$\square$}\par}
\newtheorem*{rmk}{\indent Remark}
\newenvironment*{prf}{\par\indent\textbf{\textit{Proof. }}}{\hfill $\blacksquare$\par}
\newtheorem*{ext}{\indent Extension}

\linespread{1.25}


\title{Emory University\\\textbf{MATH 250 - Foundations of Mathematics Learning Notes}}
\author{Jiuru Lyu}
\date{\today}

\linespread{1.25}

\def\Z{{\mathbb{Z}}}
\def\R{{\mathbb{R}}}
\def\C{{\mathbb{C}}}
\def\Q{{\mathbb{Q}}}
\def\E{{\mathbb{E}}}
\def\d{{\mathrm{d}}}
\def\i{{\mathrm{i}}}
\def\RE{{\mathrm{Re}}}
\def\IM{{\mathrm{Im}}}
\def\Arg{{\mathrm{Arg}}}
\def\cis{\mathrm{cis}}
\def\Proj{\mathrm{Proj}}
\def\qed{\rightline{$\blacksquare$}}
\def\ddx{\frac{\d}{\d x}}
\def\dydx{\frac{\d y}{\d x}}
\def\dx{\d x}
\def\vecx{\vec{x}}
\def\vecy{\vec{y}}
\def\vecv{\vec{v}}
\def\vecw{\vec{w}}
\def\vecu{\vec{u}}
\def\veca{\vec{a}}
\def\vecb{\vec{b}}
\def\vece{\vec{e}}
\def\DNE{\mathrm{D.N.E.}}
\def\LI{\mathrm{L.I.}}
\def\st{\ \emph{s.t.}\ }

\begin{document}
\maketitle
\tableofcontents
\newpage

\section*{Preface}
These is my personal notes for Emory University MATH 250 Foundations of Mathematics course. 

This course requires Calculus II as pre-requisite. This course focuses on Mathematical proofs and lays foundation for any other higher level math courses, such as Real Analysis, Complex Variables, Abstract Vector Space, and Abstract Algebra. The book used for this course is \textit{An Introduction to Abstract Mathematics} by Robert Bond. 

Throughout this personal note, I use different formats to differentiate different contents, including definitions, theorems, proofs, examples, extensions, and remarks. To be more specific: 
\begin{df}[Terminology]
    This is a \textbf{definition}.	
\end{df}
\begin{thm}[Theorem Name]
    This is a \textbf{theorem}.	
\end{thm}
\begin{eg}
    This is  an \textbf{example}. 
\end{eg}

\begin{sol}
    This is the \textit{answer} part of an \textbf{example}. 
\end{sol}
\begin{rmk}
	This is a \textbf{remark} of a definition, theorem, example, or proof. 
\end{rmk}

\begin{prf}
	This is a \textbf{proof} of a theorem. 
\end{prf}
\begin{ext}
	This is a \textbf{extension} of a theorem, proof, or example. 	
\end{ext}
This is a hard course, and practice will make critical thinking, mathematical thinking, and mathematical proof skills better. Even though I put efforts into making as few flaws as possible when encoding these learning notes, some errors may still exist in this note. If you find any, please contact me via email: \url{lvjiuru@hotmail.com}. 

I hope you will find my notes helpful when developing mathematical thinking skills.

\rightline{Cheers,}
\rightline{Jiuru Lyu}

\newpage
\section{Mathematical Reasoning}
\subsection{Statement}
\begin{df}[Statement]
	A \textbf{statement} is any declarative sentence that is \underline{either true or false}.
	\begin{rmk}
		A statement cannot be ambiguous, cannot be both true and false, and cannot be sometimes true or false.
	\end{rmk}
	\begin{eg}
		Examples and Non-Examples: 
		\begin{itemize}
			\item All integers are rational numbers. -- True statement
			\item $\pi$ is irrational. -- True statement
			\item $1=0.$ -- False statement
			\item $\sqrt{2}\in\Z.$ -- False statement
			\item Every student in this class is a math major. -- False statement. (To prove this false, find a student that is not a math major.)
			\item Solve the equation $2x=3.$ -- Not a statement
			\item Chocolate chip is the best ice cream flavor. -- Not a statement
			\item $x+5=3.$ -- Not a statement. (Turn it into a statement: When $x=1$, $x+5$=3; or when $x\neq-2,\ x+5\neq3.$)
		\end{itemize}
		\begin{rmk}
			$\in$ means belongs to, and $\Z$ is the notation for integers.	
		\end{rmk}
	\end{eg}
\end{df}
\begin{rmk} 
	We denote statements by letters. 
	\begin{eg}
		$P$: ``Today is a sunny day;'' $Q$: ``$3$ is a prime number.	
	\end{eg}
	If the statement's truth depends on a variable, we include the variable in the notation.
	\begin{eg}
		$R(x)$: ``$x$ is an integer;'' $P(x)$: $x+5=3$. $P(x)$ where $x\neq-2$ is false; $P(-2)$ is true.	
	\end{eg}
	\begin{eg}
		$R(f)$: ``$f$ is an increasing function.''
		\begin{itemize}
			\item If $f(x)=e^x$, then $R(f)$ is true.
			\item If $f(x)=x^2$, then $R(f)$ is false.
		\end{itemize}	
	\end{eg}
\end{rmk}
\begin{df}[Quantifiers]
	We use the symbol $\forall$ for ``for all'' or ``for any'' and the symbol $\exists$ for ``there exists.'' $\forall$ and $\exists$ are called \textbf{quantifiers}. $\forall$ is the \textbf{universal quantifier} and $\exists$ is the \textbf{existential quantifier}.	 
	\begin{eg}
		\begin{enumerate}
			\item For any $\varepsilon>0$, there is a $\delta>0$.\[\forall\varepsilon>0,\ \exists\delta>0.\]
			\item The square of every real number is non-negative. \[\forall x\in\R,\ x^2\geq0.\]
			\item There is a real number whose square is negative. \[\exists x\in\R\st x^2<0.\]
			\item For every real number $x$, there is an integer $n$ such that $n>x$. \[\forall x\in\R,\ \exists n\in\Z\st n>x.\]
			\item For all rational numbers $x$, there exist integers $a$ and $b$ such that $x=\dfrac{a}{b}$. \[\forall x\in\Q,\ \exists a,b\in\Z\st x=\frac{a}{b}.\]
		\end{enumerate}	
	\end{eg}
	\begin{eg}
		\begin{enumerate}
			\item $\forall n\in\Z,\ \exists m\in\Z,\st m=n+5$: For every integers $n$, there exists integers $m$ such that $m=n+5$.
			\item $\forall\varepsilon>0,\ \exists\delta>0,\st\forall x,y\in\R,\text{ we have }|x-y|<\varepsilon\ \Longrightarrow\ |x^2-y^2|<\delta$: For every $\varepsilon$ greater than $0$, there exists a $\delta$ greater than $0$ such that for all real $x$ and $y$ with $|x-y|$ les than $\varepsilon$, we have $|x^2-y^2|$ is less than $\delta$. 
		\end{enumerate}
	\end{eg}
	\begin{rmk}
		$\forall\varepsilon\exists\delta$ and $\exists\delta\forall\varepsilon$ are not the same.
		
		In $\forall\varepsilon\exists\delta$, $\delta$ depends on $\varepsilon$, but $\delta$ is independent in $\exists\delta\forall\varepsilon$, meaning the same $\delta$ works for all $\varepsilon$.
		\begin{eg} 
			Compare $\forall n\in\Z,\ \exists m\in\Z\st m=n^2$ to  $\exists n\in\Z\st\forall m\in\Z,\ m=n^2.$
			\begin{sol}
				\begin{enumerate}
					\item $\forall n\in\Z,\ \exists m\in\Z\st m=n^2$: True. The square of an integer is an integer. 
					\item $\exists n\in\Z\st\forall m\in\Z,\ m=n^2.$: False: $n^2$ is just one integer and cannot be represented by all $m\in\Z$.
				\end{enumerate}
			\end{sol}
		\end{eg}
	\end{rmk}
\end{df}
\begin{df}[Negations]
	If $P$ is a statement, the \textbf{negation} of $P$, written as $\neg P$ (read as ``not $P$'' or ``negation of $P$'') is the statement ``$P$ is false.''	
	\[\begin{cases}P\text{ is true}\\\neg P\text{ is false}\end{cases}\qquad\begin{cases}P\text{ is false}\\\neg P\text{ is true}\end{cases}\]
\end{df}
\begin{eg}
	Write the negation of $P$: ``All apples are fruits'' and check the truth value of both $P$ and $\neg P$.
	\begin{sol}
		$\neg P$: Not all apples are fruits / Some apples are not fruits / There exists an apple that is not a fruit.
		
		$P$ is true and $\neg P$ is false. 
	\end{sol}
\end{eg}
\begin{eg}
	Write the negation of $P$: ``Everyday this week was hot.''
	\begin{sol}
		$\neg P$: Somedays this week were not hot. / Not all days this week were hot. / There was a day this week that was not hot. 
	\end{sol}
\end{eg}
\begin{eg}
	$P$ all primes are odd.
	
	$\neg P$: There exists ($\exists$) a prime that is even / Some primes are not odd. (True: 2 is even.)
\end{eg}
\begin{eg}
	$Q$: $\forall x\in\R,\ x^2>x$
	
	$\neg Q$: $\exists x\in\R,\st x^2\leq x$ (True: for $x=\dfrac{1}{2},\ x^2=\frac{1}{4}<x$)	
\end{eg}
\begin{eg}
	$R$: There exists a real solution to $x^2+1=0$ / $\exists x\in\R\st x^2+1=0$
	
	$\neg R$: $\forall x\in\R,\ x^2+1\neq0$ -- (True	.)
\end{eg}
\begin{rmk}[Negating Quantifiers]
	In general: 
	\[\neg(\forall x\in S, P(x))=\exists x\in S\st\neg P(x). \]
	\[\neg(\exists x\in S, P(x))=\forall x\in S, \neg P(x). \]	
\end{rmk}
\begin{eg}
	$\exists n>0\st\forall m>0, m<n$
	
	Negation: $\forall n>0, \exists m>0\st m\geq n.$	
\end{eg}
\begin{eg}
	For all $x\in\R$, there exists a $y\in\R$,$\st xy=1$
	
	Negation: $\exists x\in\R\st\forall y\in\R,\ xy\neq1$	
\end{eg}

\subsection{Compound Statements}
\begin{df}[Conjunction and Disjunction]
	The \textbf{conjunction} of $P$ and $Q$, denoted $P\wedge Q$	, is the statement ``Both $P$ and $Q$ are true.'' The \textbf{disjunction} of $P$ and $Q$, denoted $P\vee Q$, is the statement ``$P$ is true or $Q$ is true.''
\end{df}
\begin{eg}
	Which of the following are true statements?
	\begin{enumerate}
	\item The function $x^2$ is even and it is concave up.
		
		$P$: $x^2$ is even (True), $Q$: $x^2$ is concave up (True).
		\[\therefore P\wedge Q\text{ is true.}\]
	\item Every student in this class is a math major and a human being.
		
		$P$: Every student in this class is a math major (False), $Q$: Every student in this class is a human being (True).
		\[\therefore P\wedge Q\text{ is false.}\]
	\item Every student in this class is a math major or a human being.
		\[P\vee Q\text{ is true.}\]
	\end{enumerate}
\end{eg}
\begin{eg}
	The following inequality can be written in conjunctions or disjunctions.
	\begin{enumerate}
		\item $|x|<3$: ($-3<x<3$) Conjunction: $x>-3$ AND $x<3$
		\item $|x|>3$: Disjunction: $x<-3$ OR $x>3$	
	\end{enumerate}
\end{eg}
\begin{rmk}
	$\vee$ is \textbf{inclusive or}, meaning if both $P$ and $Q$ are true, then $P\vee Q$ is also true, as opposed to \textbf{exclusive or}, meaning ``either $P$ or $Q$ but not both.''	
\end{rmk}
\begin{df}[Truth Tables]
	\textbf{Truth tables} are a handy way to organize information. 
\end{df}
\begin{eg}
	Write the truth table for the statement form: $P\vee\neg Q$.
	\begin{sol}\begin{center}\begin{tabular}{c|c|c|c}
		$P$&$Q$&$\neg Q$&$P\vee\neg Q$\\
		\hline
		T&T&F&T\\
		T&F&T&T\\
		F&T&F&F\\
		F&F&T&T
	\end{tabular}\end{center}\end{sol}
\end{eg}
\begin{df}[Logically Equivalent]
	Suppose $P$ and $Q$ are statements. We say $P$ and $Q$ are \textbf{logically equivalent}, denoted as $P\equiv Q$, if they are either both true or both false.
	
	Note: If two statements are logically equivalent, their truth values match up line by line in a truth table. We use this techniques to prove two statements mathematically (logically) equivalent. 
\end{df}
\begin{eg}
	Proving the following using truth tables.
	\begin{enumerate}
		\item $\neg(\neg P)\equiv P.$
		\begin{sol}\begin{center}\begin{tabular}{c|c|c}
		$P$&$\neg P$&$\neg(\neg P)$\\
		\hline
		T&F&T\\
		F&T&F
		\end{tabular}\end{center}\end{sol}
		\item $P\vee Q\equiv Q\vee P.$
		\begin{sol}\begin{center}\begin{tabular}{c|c|c|c}
			$P$&$Q$&$P\vee Q$&$Q\vee P$\\
			\hline
			T&T&T&T\\
			T&F&T&T\\
			F&T&T&T\\
			F&F&F&F
		\end{tabular}\end{center}\end{sol}
		\item $\neg(P\vee Q)\equiv(\neg P)\wedge(\neg Q).$
		\begin{sol}\begin{center}\begin{tabular}{c|c|c|c|c|c|c}
			$P$&$Q$&$\neg P$&$\neg Q$&$P\vee Q$&$\neg(P\vee Q)$&$(\neg P)\wedge(\neg Q)$\\
			\hline
			T&T&F&F&T&F&F\\
			T&F&F&T&T&F&F\\
			F&T&T&F&T&F&F\\
			F&F&T&T&F&T&T
		\end{tabular}\end{center}\end{sol}
		\item $P\vee(Q\vee R)\equiv(P\vee Q)\vee R.$
		\begin{sol}\begin{center}\begin{tabular}{c|c|c|c|c|c|c}
			$P$&$Q$&$R$&$Q\vee R$&$P\vee(Q\vee R)$&$P\vee Q$&$(P\vee Q)\vee R$\\
			\hline
			T&T&T&T&T&T&T\\
			T&T&F&T&T&T&T\\
			T&F&T&T&T&T&T\\
			T&F&F&F&T&T&T\\
			F&T&T&T&T&T&T\\
			F&T&F&T&T&T&T\\
			F&F&T&T&T&T&T\\
			F&F&F&F&F&F&F
		\end{tabular}\end{center}\end{sol}
	\end{enumerate}	
\end{eg}
\begin{thm}[Negating Compound Statements]
	\[\neg(P\wedge Q)=\neg P\vee\neg Q\]	
	\[\neg(P\vee Q)=\neg P\wedge\neg Q\]
\end{thm}
\begin{eg}
	It is not true that the numbers $x$ and $y$ are both even. 
	
	$P$: $x$ is even; $Q$: $y$ is even.
	
	$\neg(P\wedge Q)$: $x$ is odd ($\neg P$) OR $y$ is odd ($\neg Q$).	
\end{eg}
\begin{eg}
	It is not true that $9$ is prime or $9$ is even. 
	
	$P$: $9$ is prime; $Q$: $9$ is even.
	
	$\neg(P\vee Q)$: $9$ is not prime ($\neg P$) AND $9$ is odd ($\neg Q$)	
\end{eg}






















\end{document}