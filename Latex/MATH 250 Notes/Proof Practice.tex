\documentclass[10pt,letter]{article}
\usepackage{amssymb,amsmath,amsthm,amsfonts,epsfig,graphicx,dsfont,bbm, bbold, url, color, setspace, multirow,pinlabel, pifont, framed}
\usepackage[all]{xy}
\usepackage{fancyhdr, geometry}

% Page Setting
\geometry{left=2cm, right=2cm, bottom=2cm, top=2cm}
\pagestyle{fancy}
\linespread{1.75}

% Head & Foot Setting
\fancyhead[LO,LE]{} 
\fancyhead[RO,RE]{}
\chead{\textbf{Practice Proofs}} 
\cfoot{\textbf{Page \thepage\ of \pageref{LastPage}}}
\fancyfoot[LO,LE]{} 
\fancyfoot[RO,RE]{} 
\renewcommand{\headrulewidth}{0.5pt}
%\renewcommand{\footrulewidth}{0.5pt}
\parindent 2.5em

% Defining Environments
\newcounter{nq}[section]
\setcounter{nq}{0}
\newcounter{np}[section]
\setcounter{np}{0}
\newtheorem{thm}{Theorem}[section]
\newtheorem{df}{Definition}[section]
\newtheorem{eg}{Example}
\newtheorem{ax}{Axiom}[section]
\newtheorem{prop}{Proposition}[section]
\newtheorem{lem}{Lemma}[section]
\newtheorem{cor}{corollary}[section]
\newtheorem*{rmk}{\indent Remark}
\newtheorem*{ext}{\indent Extension}
\newenvironment*{p}{\par\noindent\textbf{\textit{Proof \stepcounter{np}\thenp. }}\par}{\par\hfill $\blacksquare$\par}
\newenvironment*{q}[1]{\noindent\emph{\thesection.\stepcounter{nq}\thenq$\quad $ #1}\par\noindent\texttt}{}


% Defining Symbols
\def\Z{{\mathbb{Z}}}
\def\R{{\mathbb{R}}}
\def\C{{\mathbb{C}}}
\def\Q{{\mathbb{Q}}}
\def\N{{\mathbb{N}}}
\def\E{{\mathbb{E}}}
\def\d{{\mathrm{d}}}
\def\i{{\mathrm{i}}}
\def\RE{{\mathrm{Re}}}
\def\IM{{\mathrm{Im}}}
\def\Arg{{\mathrm{Arg}}}
\def\cis{\mathrm{cis}}
\def\Proj{\mathrm{Proj}}
\def\qed{\rightline{$\blacksquare$}}
\def\ddx{\frac{\d}{\d x}}
\def\dydx{\frac{\d y}{\d x}}
\def\dx{\d x}
\def\vecx{\vec{x}}
\def\vecy{\vec{y}}
\def\vecv{\vec{v}}
\def\vecw{\vec{w}}
\def\vecu{\vec{u}}
\def\veca{\vec{a}}
\def\vecb{\vec{b}}
\def\vece{\vec{e}}
\def\DNE{\mathrm{D.N.E.}}
\def\LI{\mathrm{L.I.}}
\def\st{\emph{ s.t. }}
\def\fs{\emph{ f.s. }}
\def\Iff{\emph{ iff }}

% Title Information
\title{Foundations of Mathematics -- \textbf{Proof Practice}}
\author{Jiuru Lyu}
\date{\today}

\begin{document}


\section{Statements}
\begin{framed}\begin{q}
	{Class Handout, Chapter 1.3, Implications.}
	{Let $a,\ b,\ $and $c$ be integers, with $a$ and $b$ non-zero. If $(ab)\mid(ac)$, then $b\mid c.$}
\end{q}\end{framed}
\begin{p}
	Let	$a,b,c\in\Z$ with $a\neq0$ and $b\neq0$.\par Suppose $(ab)\mid(ac)$. Then $\exists k\in\Z\st ac=(ab)k$.\par Divide both sides of the equation by $a$: \[c=bk.\] \par Since $k\in\Z$, by definition of divides, $b\mid c$.
\end{p}

\begin{framed}\begin{q}
	{Class Handout, Chapter 1.4, Contrapositive and Converse}
	{Prove that for all real numbers $a$ and $b$, if $a\in\Q$ and $ab\notin\Q$, then $n\notin Q$.}	
\end{q}\end{framed}
\begin{p}
	Let $a,b\in\Q$.\par Assume for the sake of contradiction that if $a\in\Q$ and $ab\notin\Q$, we have $b\in\Q$.\par Then, $\exists p,q,m,n\in\Z\st a=\dfrac{m}{n}$ and $b=\dfrac{p}{q}$.\par Hence, \[ab=\frac{m}{n}\cdot\frac{p}{q}=\frac{mp}{nq}\]\par As $mp, nq\in\Z$, $ab\in\Q.$\begin{center}$\divideontimes$ This contradicts with the fact that $ab\notin\Q$.\end{center}\par So, $b$ must not be rational. 
\end{p}

\begin{framed}\begin{q}
{Chapter 1.1 \# 7(c)}
{Prove the square of an even integer is divisible by 4.}	
\end{q}\end{framed}
\begin{p}
	Suppose $x\in\Z$ is even. Then $\exists k\in\Z\st x=2k.$\par Then, $x^2=(2k)^2=4k^2.$\par Since $k^2\in\Z$, we have $4\mid 4k^2$.
\end{p}

\begin{thm}[Archimedean Principle]\label{AP}
	For every real number $x$, there is an integer $n$, such that $n>x$.	
\end{thm}

\begin{framed}\begin{q}
	{Chapter 1.1 \# 11}
	{For every positive real number $\varepsilon$, there exists a positive integer $N$ such that $\dfrac{1}{n}<\varepsilon$ for all $n\geq N.$}
\end{q}\end{framed}
\begin{p}
	Suppose $\varepsilon\in\R$ and $\varepsilon>0.$\marginpar{}\par Since $\varepsilon\in\R$, we have $\dfrac{1}{n}\in\R.$\par Then, by Archimedean Principle, $\exists n\in\Z\st n>\dfrac{1}{\varepsilon}.$\par Hence, $n\varepsilon>1$ or $\varepsilon>\dfrac{1}{n}.$\par Suppose $N\in\Z\st N=\left\lceil\dfrac{1}{\varepsilon}\right\rceil$\par\textit{$\left\lceil\dfrac{1}{\varepsilon}\right\rceil$ means the integer greater to $\dfrac{1}{\varepsilon}$ if $\dfrac{1}{\varepsilon}\notin\Z$, and the integer equals to $\dfrac{1}{\varepsilon}$ if $\dfrac{1}{\varepsilon}\in\Z.$}\par Hence, $N\geq\dfrac{1}{\varepsilon}.$\par As $n>\dfrac{1}{\varepsilon},$ we have $n\geq N$
\end{p}

\begin{framed}\begin{q}
	{Chapter 1.1 \# 12}
	{Use the Archimedean Principle (Theorem \ref{AP}) to prove if $x$ is a real number, then there exists a positive integer $n$ such that $-n<x<n.$}
\end{q}\end{framed}
\begin{p}
	Suppose $x\in\R.$\par $\boxed{\text{Case }1}$ If $x>0$, then $-x<0$ (i.e., $-x<0<x$).\par\hspace{10mm} By the Archimedean Principle, $\exists n\in\Z\st n>x.$\par\hspace{10mm} Multiply $(-1)$ on both sides of the inequality: \[-n<-x\]\par\hspace{10mm} As $-x<0<x$, \[-n<-x<0<x<n,\]\par\hspace{10mm} which means $-n<x<n,$ and $n$ is positive.\par $\boxed{\text{Case }2}$ If $x<0,$ then $-x>0$ (i.e., $-x>0>x$)\par\hspace{10mm} Since $x\in\R$, we have $-x\in\R$.\par\hspace{10mm} By the Archimedean Principle, $\exists n\in\Z\st n>-x.$\par\hspace{10mm} Multiply $(-1)$ on both sides of the inequality: \[-n<x\]\par\hspace{10mm} As $x<0<-x,$ \[-n<x<0<-x<n,\]\par\hspace{10mm} which means $-n<x<n,$ and $n$ is positive.\par In all cases, we have proven that $x\in\R\implies\exists n\in\Z,\ n>0\st-n<x<n.$
\end{p}

\begin{framed}\begin{q}
	{Chapter 1.1 \# 13}
	{Prove that if $x$ is a positive real number, then there exists a positive integer $n$ such that $\dfrac{1}{n}<x<n$.}
\end{q}\end{framed}
\begin{p}
	Suppose $x\in\R,\ x>0$\par $\boxed{\text{Case }1}$ If $0<x\leq1,$ then $\dfrac{1}{x}\geq1.$\par\hspace{10mm} Hence, $x\leq1\leq\dfrac{1}{x}.$\par\hspace{10mm} As $x\in\R,$ $\dfrac{1}{x}\in\R$, then by the Archimedean Principle (Theorem \ref{AP}): \[\exists n\in\Z\st n>\dfrac{1}{x}.\]\par\hspace{10mm} Hence, $nx>1$ or $x>\dfrac{1}{n}.$\par\hspace{10mm}As $x\leq\dfrac{1}{x},$ $n>\dfrac{1}{x}$, and $x>\dfrac{1}{n}$, we have \[\frac{1}{n}<x<n.\]\par$\boxed{\text{Case }2}$ If $x>1$, then $0<\dfrac{1}{x}<1.$\par\hspace{10mm} Hence, $\dfrac{1}{x}<1<x.$\par\hspace{10mm} As $x\in\R$, by the Archimedean Principle: \[\exists n\in\Z\st n>x>0\]\par\hspace{10mm} Hence, $\dfrac{1}{n}<\dfrac{1}{x}$\par\hspace{10mm} As $\dfrac{1}{x}<x$, $\dfrac{1}{n}<\dfrac{1}{x},$ and $n>x,$ we have \[\dfrac{1}{n}<x<n\]\par In all cases, we proven that $x\in\R,\ x>0\implies\exists n\in\Z, n>0\st\dfrac{1}{n}<x<n.$
\end{p}

\begin{framed}\begin{q}
	{Handout Chapter 1.4-2 More Contradictions and Equivelance}
	{There are no positive integer solutions to the equation $x^2-y^2=10$.}
\end{q}\end{framed}
\begin{p}
	Assume for the sake of contradiction that there are positive integer solutions to the equation $x^2-y^2=10.$\par Suppose $\exists x,y\in\Z$ and $x>0,\ y>0\st x^2-y^2=10.$\par Then, we have $x^2=10+y^2.$\par Since $x>0,\ x^2>0,$ we have $10+y^2>0.$\par Then, $y^2>-10.$\par\begin{center}$\divideontimes$ This contradicts with the fact that $y^2\geq0$ if $y\in\Z.$\end{center}\par So, our assumption is wrong. There must be no positive integer solutions to the equation $x^2-y^2=10.$
\end{p}

\begin{framed}\begin{q}
	{Handout Chapter 1.4-2 More Contradictions and Equivelance}
	{Show that if $a\in\Q$ and $b\in\Q'$, then $a+b\in\Q'$}
	\begin{rmk}
		The notation $\Q$ means the set for rational numbers, and $\Q'$ means the set for irrational numbers.
	\end{rmk}
\end{q}\end{framed}
\begin{p}
	Suppose $a\in\Q$ and $b\in\Q'$\par Assume for the sake of contradiction that $a+b\in\Q$.\par Then, $\exists m,n,p,q\in\Z$ such that $a=\dfrac{m}{n}$ and $a+b=\dfrac{p}{q}.$\par Then, \[b=\frac{p}{q}-a=\frac{p}{q}-\frac{m}{n}=\frac{pn-mq}{qn}\in\Q\]\par Since $pn-mq\in\Q$ and $qn\in\Z$, we have $b=\dfrac{pn-mq}{qn}\in\Q$.\par\begin{center}$\divideontimes$ This contradicts with the fact that $b\in\Q'$. \end{center}\par So, $a+b$ must be irrational. 
\end{p}

\begin{framed}\begin{q}
	{Handout Chapter 1.4-2 More Contradictions and Equivelance}
	{If $n\in\N$ and $2^n-1$ is prime, then $n$ is prime.}
\end{q}\end{framed}
\begin{p}
	We will prove the contrapositive: if $n$ is not prime, then $2^n-1$ is not prime.\par Suppose $n$ is not prime. Then, $\exists a,b\in\Z$ with $1<a,b<n\st n=ab.$\par Then, $2^n-1=2^{ab}=\left(2^a\right)^b-1.$\par Notice that for $x^w-1$, by polynomial long division, have \[x^w-1=(x-1)\left(x^{w-1}+x^{w-2}+\cdots+1\right),\]\par Substitute $x=2^a$ and $w=b$, we have \[2^n-1=\left(2^a-1\right)\left[\left(2^a\right)^{b-1}+\left(2^a\right)^{b-2}+\cdots+1\right].\]\par Since $\left(2^a-1\right)\in\Z$ and $\left[\left(2^a\right)^{b-1}+\left(2^a\right)^{b-2}+\cdots+1\right]\in\Z,$ we see that $2^n-1$ is not prime. 
\end{p}











\label{LastPage}
\end{document}